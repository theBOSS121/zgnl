\documentclass{article}

\usepackage{tikz}
\usepackage{verbatim}
\usepackage{parskip}
\usepackage{amsthm}
\usepackage{xpatch}
\usepackage{amsmath}
\usepackage{graphicx}

\graphicspath{ {./img/} }

\setlength\parindent{0pt}

\newtheorem{definicija}{Definicija}[subsection]
\newtheorem{lema}{Lema}[subsection]
\newtheorem{izrek}{Izrek}[subsection]
\newtheorem{trditev}{Trditev}[subsection]
\newtheorem{posledica}{Posledica}[subsection]
\newtheorem{domneva}{Domneva}[subsection]
\newtheorem{primer}{Primer}[subsection]
\newtheorem{opomba}{Opomba}[subsection]

\makeatletter
\xpatchcmd{\@thm}{\thm@headpunct{.}}{\thm@headpunct{:}}{}{}
\makeatother

\begin{document}
\pagestyle{empty}

\begin{comment}
definitions
\end{comment}

%%%%%%%%%%%%%%%%%%%%%%%%%%%%%%%%%%%%%%%%%%%%%%%%%%%%%%%%%%%%%%%%%
\section{ Trikotniki in štirikotniki }
\subsection{ Trikotniki }

\begin{definicija}[Trikotnik]
    Trikotnik je geometrijski lik, ki je določen s tremi točkami, ki ne ležijo na isti premici. 
\end{definicija}


\end{document}