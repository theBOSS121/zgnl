\documentclass{article}

\usepackage{tikz}
\usepackage{verbatim}
\usepackage{parskip}
\usepackage{amsthm}
\usepackage{xpatch}
\usepackage{amsmath}
\usepackage{graphicx}

\graphicspath{ {./img/} }

\setlength\parindent{0pt}

\newtheorem{definicija}{Definicija}[subsection]
\newtheorem{lema}{Lema}[subsection]
\newtheorem{izrek}{Izrek}[subsection]
\newtheorem{trditev}{Trditev}[subsection]
\newtheorem{posledica}{Posledica}[subsection]
\newtheorem{domneva}{Domneva}[subsection]
\newtheorem{primer}{Primer}[subsection]
\newtheorem{opomba}{Opomba}[subsection]

\makeatletter
\xpatchcmd{\@thm}{\thm@headpunct{.}}{\thm@headpunct{:}}{}{}
\makeatother

\begin{document}
\pagestyle{empty}

\begin{comment}
definitions
\end{comment}

%%%%%%%%%%%%%%%%%%%%%%%%%%%%%%%%%%%%%%%%%%%%%%%%%%%%%%%%%%%%%%%%%
\section{ Obseg in ploscina }
\subsection{ Večkotniki }

\begin{definicija}[Obseg]
    Obseg večkotnika je vsota dolžin njegovih stranic.
\end{definicija}

\begin{definicija}[Ploščina]
    Ploščina večkotnika je število vseh ploščinskih enot, s katerimi je lik prekrit, ali pa vsota ploščin vseh pravokotnikov in kvadratov, na katere je večkotnik razdeljen.
\end{definicija}

\pagebreak
\subsection{ Paralelogram }

\begin{trditev}[Obseg paralelograma]
    Obseg paralelograma je vsota dolžin vseh njegovih stranic.
    \[ o = 2 \cdot a + 2 \cdot b \]
\end{trditev}

\begin{trditev}[Ploščina paralelograma]
    Ploščina paralelograma je enaka produktu dolžine stranice in pripadajoče višine.
    \[ p = a \cdot v_a \quad \text{ali} \quad p = b \cdot v_b \]
\end{trditev}

\begin{trditev}[Obseg romba]
    Obseg romba je štirikratnik dolžine stranice.
    \[ o = 4 \cdot a \]
\end{trditev}

\begin{trditev}[Ploščina romba]
    Ploščina romba je enaka produktu dolžine stranice in višine.
    \[ p = a \cdot v_a \]
\end{trditev}


\end{document}